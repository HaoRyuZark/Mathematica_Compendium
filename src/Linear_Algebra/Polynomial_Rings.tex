\newpage
\section{Polynomial Rings}

Given a field \(\Field\) with \(\Field^{\Integers_{\ge 0}}\) we define the multiplication 

\[
    (a_0, a_1, \dots ) \cdot (b_0, b_1, \dots ) := (c_0, c_1, \dots) 
\]

Where 

\[
    c_0 := a_0 b_0, \quad c_1:= a_0 b_1 + a_1 b_0 \quad c_2 := a_0 b_2 + a_1 b_1 + a_2 b_0
\]

In general for \(n \ge 0\)

\[
    c_n := \sum_{i = 0}^{n} a_i b{n - i} = a_0 b_{n} + a_1 b_{n - 1} + cdots + a_n b_0
\]

And the addition as 

\[
    (a_i)_{i \in \Naturals} + (a_i)_{i \in \Naturals} = (a_i + b_i)_{i \in \Naturals}
\]

This \((\Field^{\Integers_{\ge 0}}, +, \cdot)\) is also noted as \(\Field[[\Field]]\) and 
called the \emph{power-ring} over \(\Field\). Also, instead of \((a_0, a_1, \dots)\) we write 

\[
    f = \sum_{i=0}^{\infty} a_i X^i
\]

For \(X := (0, 1, 0, 0, \dots)\). We call this ring also the \emph{formal power series} over \(\Field\).  

The members \(f = (a_0, a_1, \dots, ) \in \Field [[X]]\) are called \emph{polynomials}

For the finite case we write 

\[
    f = \sum_{i=0}^{n} a_i X^i
\]

The set of all polynomials reachable via the multiplication vector-space properties of \(\Field[[X]]\) is 
called the \emph{polynomial ring} \(\Field[X]\). The field before the \([X]\) tells us which elements 
are allowed as coefficients.

\textbf{Example:}

For \(f := (1,2,3,4, \dots) = 1 + 2X + 3X^2 + 4X^3 \)

\[
    f \cdot 4 = (4,\,1,\,5,\,2,\,0,\,0,\,\ldots) = 4 + 1X + 5X^{2} + 2X^{3}
\]

\[
    f \cdot 3X = (0,\,3,\,6,\,2,\,5,\,0,\,0,\,\ldots) = 3X + 6X^{2} + 2X^{3} + 5X^{4}
\]

\[
    f \cdot 2X^{2} = (0,\,0,\,2,\,4,\,6,\,1,\,0,\,0,\,\ldots) = 2X^{2} + 4X^{3} + 6X^{4} + X^{5}
\]

\[
    f \cdot X^{3} = (0,\,0,\,0,\,1,\,2,\,3,\,4,\,0,\,0,\,\ldots) = X^{3} + 2X^{4} + 3X^{5} + 4X^{6}
\]

\subsection{Degree of a Polynomial}

The \emph{degree} of a polynomial is given by the \(n\) in \(\sum_{i = 0}^{n} a_i X^i\) with \(f \ne \vec{0}\).
Also, note that we define \(\text{def}(0):= -\infty\)

\emph{Degree Addition}

For \(f,g \in \Field[X] \setminus \{0\}\) the following holds 

\[
    \deg(f + g) \le \max(\deg(f),\deg(g))
\]

\emph{Degree Multiplication}

For \(f,g \in \Field[X] \setminus \{0\}\) the following holds 

\[
    \deg(fg) \le \max()\deg(f) + \deg(g)
\]

\subsection{Properties of the Polynomial Ring}

\emph{Neutral Element}

\[
    1 := (1, 0, 0, \dots) \in \Field[X]
\]

is the neutral element of the multiplication in \(\Field[[X]]\) and \(\Field[X]\).

\emph{\(X_a\)}

\[
    X_a = (0, a_0, a_1, \dots) \forall a \in \Field [[X]]
\]

\emph{Multiplication}

\[
    X^i X^j = X^{i + j}, \text{ with } X^0 := 1
\]

\emph{Identity}

Given \(f \in \Field[X]\) with degree \(n\) then 

\[
    f = a_0 + a_1 X + \cdots + a_n X^n
\]

gives a clear identification of \(\Field\) with 

\[
    \{0\} \cup \{f \in \Field [X] \mid \deg(f) = 0\}
\]

\emph{Vector space}

\[
    \Field[X]_{\deg < n} := \{0\} \cup \{f \in \Field [X] \mid \deg(f) < n\}
\]

is a vector space.


\subsection{Field Algebra}

Given a ring \(R\) with the one of the vector space with the field \(\Field\). We call 
\(R\) an \emph{associative algebra} with the one if only if 

\[
    a(rs) = (ar)s = r(as)
\]

for all \(r,s \in R\) and \(a \in \Field\).

\subsection{Lemmi of the Division}

Given the field \(\Field\) then 

\begin{itemize}

    \item  \(\Field[X]\) is a commutative ring with one and exactly one commutative \(\Field\)-Algebra with 1.

    \item For \(f, g \in \Field[X]\) with \(g \ne 0\) there are well-defined \(q,r \in \Field[X]\) with 
    
        \[
            f = qg + r \text{ and } \deg(r) < \deg(g) \lor r = 0
        \]
    
\end{itemize}

\textbf{Proof:}

For the first lemma, to show that the multiplication is commutative we just 
use the commutative property of commutativity of our field \(\Field\). 

The distributive property for \(f,g,h \in \Field [X]\) and the scalar multiplication 
are also easy to show using our definitions. We will show the associativity. 

We will use induction: If \(\deg(h) = 0\), then \(h \in \Field\) and the associativity holds because of the 
associativity for scalars. 

Let us assume that it holds for \(\deg(h) \le n\). If  \(\deg(h) > 0 \) then

\[
    h = \hat{h} + a_{n + 1}X^{n + 1} \text{ with } \hat{h} \in \Field [X]
\]

is a polynomial of degree at most \(n\) or 0. Due to that fact, \((fg)\hat{h} = f(g\hat{h})\) 

Induction step:

\begin{align*}
    (fg)h &= (fg)(\hat{h} + a_{n + 1}X^{n + 1}) \\  
          &= fg\hat{h} + fga_{n + 1}X^{n + 1}) \\ 
          &= fg\hat{h} + f(ga_{n + 1)}X^{n + 1}) \\ 
          &= f(g\hat{h} + ga_{n + 1}X^{n + 1})) \\ 
          &= f(g(\hat{h} + a_{n + 1}X^{n + 1}))) \\ 
          &= f(gh)
\end{align*}

\QED 

As for the second lemma, given \(\deg(f) = m, \deg(g) = n, f = \sum_{i = 0}^{m} a_i X^i, g = \sum_{i = 0}^{n} b_i X^i\).
If \(f = 0\) then \(q = 0\) and \(r = 0\). If \(m < n\) we are done, because \(q = 0, r = f\). 

For the case \(m \ge n\) we will show the existence of \(q,r\) per induction using \(m\). For 
\(m = 0\) are \(n = 0\), \(f = a_0\) \(g = b_0\). Set \(q = \frac{a_0}{q_0}\) and \(r = 0\), and we get 
\(f = qg + r\).

If \(m > 0\) and we assume that for each polynomial \(h\) with \(\deg(h) le m - 1\) there 
exists polynomials \(\hat{q}, \hat{r}\) such that \(\deg(\hat{r}) < \deg(g)\) and 
\(h = \hat{q}g + \hat{r}\).

Define \(\hat{f} := f - \frac{a_m}{b_n}X^{m - n}g\). Then \(\deg(\hat{f}) \le m - 1\). There also 
exist \(\hat{q}, \hat{r}\) with \(\deg(\hat{r}) < \deg(g)\) and \(\hat{f} \hat{q}g + \hat{r}\). Then 

\[
    f = \hat{f} + \frac{a_m}{b_n} X^{m-n} g
    = \hat{q} g + \hat{r} + \frac{a_m}{b_n} X^{m-n} g
    = \left(\hat{q} + \frac{a_m}{b_n} X^{m-n}\right) g + \hat{r}
\]

Hence, with \(q = \hat q + \frac{a_m}{b_n} X^{m-n}\) and \(r = \hat{r}\), existence is shown.

For uniqueness: Let \(f = qg + r\) and \(f = \hat{q}g + \hat{r}\)
with \(\deg(r) < \deg(g)\) and \(\deg(\hat{r}) < \deg(g)\). Then it follows that

\[
    r - \hat{r} = (q - \hat{q}) g .
\]

If \(q - \hat{q} \neq 0\), then \(\deg(r - \hat{r}) \ge \deg(g)\),
which is a contradiction. Thus, \(q = \hat{q}\) and \(r = \hat{r}\).

\QED

\subsection{Greatest Common Divisor for Polynomials}

For two polynomials \(f,g \Field[X]\) with \((f,g \ne (0,0))\). The polynomial 
\(t \in \Field[X]\) with 

\begin{itemize}

    \item \(t \mid f\) and \(t \mid g\)

    \item For all \(c \in \Field[X]\) with \(c \mid f\) and \(c \mid g\) holds \(c \mid t\)

\end{itemize}

is considered the \emph{greatest common divisor} of \(f\) and \(g\). If \(t = 1\) 
then \(f,g\) are \emph{relative prime}.

\subsection{More Properties of Polynomials}

\begin{itemize}

    \item In \(\mathbb{Z}\) one has \(|ab| = |a||b|\); in \(\Field[X]\) one has 
    \(\deg(fg) = \deg(f) + \deg(g)\).

    \item The gcd of two polynomials \(f, g \in \Field[X]\) with 
    \((f,g) \neq (0,0)\) is unique only up to factors in \(\Field[X]^{\ast}\), 
    that is, up to factors in \(\Field[X]\) of degree \(0\). Notation: 
    \(t = \gcd(f,g)\) actually also denotes every multiple \(a t\) with 
    \(a \in \Field \setminus \{0\}\) as a gcd of \(f\) and \(g\).

    \item In both rings there is a division with remainder, 
    and therefore \(\Field[X]\) also has a Euclidean algorithm for computing 
    the gcd and an extended Euclidean algorithm for determining Bézout 
    coefficients. Here the degree of a polynomial plays the role of the 
    absolute value in \(\Integers\). We give the first part of the algorithm again.

    \item Furthermore, one also has the analogue of prime numbers in 
    \(\Field[X]\), namely irreducible polynomials, and correspondingly a unique prime factorization for 
    polynomials.

\end{itemize}

\subsection{Euclidean Algorithm for Polynomials}

\textbf{Input:} \(f, g \in \Field[X]\), \((f, g) \neq (0,0)\).

\textbf{Output:} The greatest common divisor \(t\) of \(f\) and \(g\).

\textbf{Algorithm:}

\begin{enumerate}

    \item If \(f = 0\), then return \(g\). If \(g = 0\), return \(f\).

    \item Set \(r_1 := f\), \(r_2 := g\), \(k := 2\).

    \item As long as \(r_k \neq 0\), define \(r_{k+1} := r_k\).

    \item As soon as \(\deg(r_k) = 0\), return \(r_{k-1}\).

\end{enumerate}

\textbf{Example:}


\subsection{Reducible Irreducible}

A polynomial \(p \in \Field[X] \setminus \{0\}\) is called \emph{irreducible} if 
\(p\) is not a unit (that is, \(p \notin \Field[X]^{\ast}\)) and if for 
\(f, g \in \Field[X]\) the following holds: 
\(p = f \cdot g \Rightarrow f \in \Field[X]^{\ast} \,\vee\, g \in \Field[X]^{\ast}\).
An element \(h \in \Field[X] \setminus \{0\}\) for which 
\(h = f \cdot g\) holds with \(f, g \in \Field[X] \setminus \Field[X]^{\ast}\) 
is called \emph{reducible}.

\subsection{Normalized}

A polynomial \(f \in \Field[X] \setminus \{0\}\) with \(f = \sum_{i = 0}^{n} a_i X^i\) is considered 
\emph{normalized} if \(a_n = 1\).

\subsection{Factorization Property}

Each polynomial \(f \in \Field[X] \setminus \{0\}\) has from to one to the order of the factors unique 
factorizations in a normalized irreducible form and a unit.