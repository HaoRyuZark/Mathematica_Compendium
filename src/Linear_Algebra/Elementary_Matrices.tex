\newpage
\section{Elementary Matrix}

We are going to define 3 matrices that can be obtained from the Elementary Row
Operations on the \emph{Identity Matrix} and call them \emph{Elementary Matrices}.

\emph{C1: Row Addition}

\[
    C_1 :=
    \begin{bmatrix}
        1 &        &         &   &        \\
          & \ddots &         &   &        \\
          &        & 1       &   &        \\
          &        & \lambda & 1 &        \\
          &        &         &   & \ddots \\
    \end{bmatrix}
    \in \Field^{n \times n}
    \quad \text{or} \quad
    C_1 :=
    \begin{bmatrix}
        1 &        &   &         &        \\
          & \ddots &   &         &        \\
          &        & 1 & \lambda &        \\
          &        &   & 1       &        \\
          &        &   &         & \ddots \\
    \end{bmatrix}
    \in \Field^{n \times n}
\]

\emph{C2: Row swap}

\[
    C_2 :=
    \begin{bmatrix}
        1 &        &   &   &        &   \\
          & \ddots &   &   &        &   \\
          &        & 0 & 1 &        &   \\
          &        & 1 & 0 &        &   \\
          &        &   &   & \ddots &   \\
          &        &   &   &        & 1 \\
    \end{bmatrix}
    \in \Field^{n \times n}
\]

\emph{C3: Row Scaling}

\[
    C_3 :=
    \begin{bmatrix}
        1 &        &         &        &   \\
          & \ddots &         &        &   \\
          &        & \lambda &        &   \\
          &        &         & \ddots &   \\
          &        &         &        & 1 \\
    \end{bmatrix}
    \in \Field^{n \times n}
\]

Each of them also encodes the row Operations from where they came from.
So now Gaussian Elimination can be encoded as matrix multiplication.

\subsection{Invertibility}

Because every row operation is invertible now we can also claim that
Elementary Matrices are invertible and thy also represented that inverted row operation.

\subsection{Determinants}

The determinants corresponding to each matrix are:

\begin{align*}
     & \det C1 = 1       \\
     & \det C2 = -1      \\
     & \det C3 = \lambda
\end{align*}

