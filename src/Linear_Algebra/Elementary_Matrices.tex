\newpage
\section{Elementary Matrix}

We are going to define 3 matrices that can be obtained from the Elementary Row
Operations on the \emph{Identity Matrix} and call them \emph{Elementary Matrices}.

\emph{C1: Row Addition}

\[
    C_1 :=
    \begin{bmatrix}
        1 &        &         &   &        \\
          & \ddots &         &   &        \\
          &        & 1       &   &        \\
          &        & \lambda & 1 &        \\
          &        &         &   & \ddots \\
    \end{bmatrix}
    \in \Field^{n \times n}
    \quad \text{or} \quad
    C_1 :=
    \begin{bmatrix}
        1 &        &   &         &        \\
          & \ddots &   &         &        \\
          &        & 1 & \lambda &        \\
          &        &   & 1       &        \\
          &        &   &         & \ddots \\
    \end{bmatrix}
    \in \Field^{n \times n}
\]

\emph{C2: Row swap}

\[
    C_2 :=
    \begin{bmatrix}
        1 &        &   &   &        &   \\
          & \ddots &   &   &        &   \\
          &        & 0 & 1 &        &   \\
          &        & 1 & 0 &        &   \\
          &        &   &   & \ddots &   \\
          &        &   &   &        & 1 \\
    \end{bmatrix}
    \in \Field^{n \times n}
\]

\emph{C3: Row Scaling}

\[
    C_3 :=
    \begin{bmatrix}
        1 &        &         &        &   \\
          & \ddots &         &        &   \\
          &        & \lambda &        &   \\
          &        &         & \ddots &   \\
          &        &         &        & 1 \\
    \end{bmatrix}
    \in \Field^{n \times n}
\]

Each of them also encodes the row Operations from where they came from.
So now Gaussian Elimination can be encoded as matrix multiplication.

Formally,

\[
    C_1:= \text{Add}_m (i, j, \lambda) := I_m + a e_i f_j: m \times m \to \Field : (s,t) \mapsto 
    \begin{cases}
        1, & s = t \\ 
        \lambda, & (s,t) = (i,j) \\ 
        0, &\text{ else} 
    \end{cases}
\]

\[
    C_2:= \text{Swap}_m (i,j) := (e_1, \dots, e_{i - 1}, e_j, e_{j + 1}, \dots, e_{j - 1}, e_i, e_{j + 1}, \dots, e_m): m \times m \to \Field : (s,t) \mapsto
    \begin{cases}
        1, & s = t \notin \{i, j\} \\ 
        1, & (s, t) \in \{(i,j), (j,i)\}\\ 
        0, &\text{ else} 
    \end{cases}
\]

\[
    C_3:= \text{Mul}_m (i,\lambda) := I_m + (a - 1) e_i f_j: m \times m \to \Field : (s,t) \mapsto 
    \begin{cases}
        1, & s = t  \ne 1 \\ 
        \lambda, & s = t = i \\ 
        0, &\text{ else} 
    \end{cases}
\]

\subsection{Invertibility}

Because every row operation is invertible now we can also claim that
Elementary Matrices are invertible and thy also represented that inverted row operation.

Formally, 

\begin{itemize}
    
    \item \(\text{Add}_m (i, j, a)\) has the inverse \(\text{Add}_m (i, j, - a)\).

    \item \(\text{Mul}_m (i,\lambda)\) has the inverse \(\text{Mul}_m (i,\lambda^-1)\).
    
    \item \(\text{Swap}_m (i,j)\) has itself as the inverse.

\end{itemize}

\subsection{Determinants}

The determinants corresponding to each matrix are:

\begin{align*}
     & \det C1 = 1       \\
     & \det C2 = -1      \\
     & \det C3 = \lambda
\end{align*}

