\newpage
\section{Continuity}

A real-valued function \(f\) is continuous if it is continuous for all points in its 
domain.

\subsection{Definition of Continuity at a Point}

A function \(f\) is \emph{continuous} at a point \(x_0\) if:

\[
    \forall \varepsilon > 0, \exists \delta > 0 \text{ such that } |x - x_0| < \delta \Rightarrow |f(x) - f(x_0)| < \varepsilon
\]

Equivalently,

\[
    \lim_{x \to x_0} f(x) = f(x_0)
\]

To solve continuity problems the general approach is to:

\begin{enumerate}

    \item Write down the definition.

    \item Substitute \(f\) for the concrete function into the definition.

    \item Modify the expression with the concrete \(f\) \(|f(x) - f(y)|\) in way that you get 
          \(| x - y|\) inside the expression.
    
    \item Choose a \(\delta\) if not given. On the one side it is often \(\delta = |x - y |\) and then 
          substitute into the expression. On the other side if \(\delta\) is given make an approximation 
          which is greater than the current expression and then simplify. In both cases after simplifying 
          you should get a \(\varepsilon\) is the function is continuous. 

\end{enumerate}

\textbf{Example:} 

\(\varepsilon-\delta\) Exercise for \(f(x) = \frac{1}{x}\) at \(x_0 > 0\)

We want to find \(\delta\) such that:

\[
    |x - x_0| < \delta \Rightarrow \left|\frac{1}{x} - \frac{1}{x_0}\right| < \varepsilon
\]

Start by rewriting:
\[
    \left|\frac{1}{x} - \frac{1}{x_0}\right| = \left|\frac{x_0 - x}{xx_0}\right| = \frac{|x - x_0|}{|x||x_0|}
\]

Assume:

\[
    x \in \left[\frac{x_0}{2}, \frac{3x_0}{2}\right] \Rightarrow |x| \ge \frac{x_0}{2}
\]

Then:

\[
    \left|\frac{1}{x} - \frac{1}{x_0}\right| \le \frac{|x - x_0|}{\frac{x_0}{2} \cdot x_0} = \frac{2}{x_0^2} |x - x_0|
\]

Choose:

\[
    \delta = \min\left\{\frac{x_0}{2}, \frac{x_0^2 \varepsilon}{2} \right\}
\]

\subsection{Characterization using sequences}

Let \(f : (a,b) \to \mathbb{R}\) be a function and \(x_0 \in (a,b\).
The function \(f\) is continuous at \(x_0\) if and only if for every sequence 
\(\{x_n\}_{n\in \Naturals}\) with \(x_n \in (a,b)\) for all \(n \in \Naturals\)
and \(\lim_{n\to\infty} x_n = x_0\), it holds that 
\(\lim_{n\to\infty} f(x_n) = f(x_0)\).

\subsection{Left and Right Continuity}

Let \( f : D \to \Reals \) be a function and \( x_0 \in D \).
The function \( f \) is called \emph{left-continuous} at \( x_0 \) if
\( f \) is continuous on \( D \cap (-\infty, x_0] \).
Analogously, \( f \) is \emph{right-continuous} at \( x_0 \) if
\( f \) is continuous on \( D \cap [x_0, \infty) \).

\subsection{Rules of Continuity}

Let \(f\) and \(g\) be continuous at \(x_0\):

\begin{itemize}
    
    \item \(f + g\), \(f - g\), \(f \cdot g\) are continuous at \(x_0\)
    
    \item \(\frac{f}{g}\) is continuous at \(x_0\) if \(g(x_0) \ne 0\)
    
    \item Compositions: if \(g\) is continuous at \(x_0\), and \(f\) is continuous at \(g(x_0)\), 
    then \(f \circ g\) is continuous at \(x_0\)

\end{itemize}

\subsection{Continuity and Compactness}

A function over an interval \([a,b]\) which is continuous is \emph{bounded}.

\textbf{Proof:}

If \(f\) is not bounded, then \(\forall n \in \Naturals, \exists x_n \in [a,b]\) such that
\[
    |f(x_n)| > n
\]

By the Bolzano-Weierstrass theorem, there exists a convergent subsequence \(\{x_{n_k}\} \subset [a,b]\) such that

\[
    \lim_{k \to \infty} x_{n_k} = x^* \in [a,b]
\]

By continuity of \(f\):

\[
    \lim_{k \to \infty} f(x_{n_k}) = f(x^*) = \infty
\]

But by construction \(|f(x_{n_k})| > n_k \to \infty\) as \(k \to \infty\), which is a contradiction to the 
\(x^* \in \mathcal{D}(f)\)

\QED

\subsection{Extreme-Value Theorem}

If \(f: [a, b] \to \Reals\) is continuous on the closed interval \([a, b]\), then exist \(x_m, x_M in [a,b]\) such that 

\[
    f(x_m) = \inf_{[a,b]} f, \quad f(x_M) = \sup_{[a,b]} f
\]

\textbf{Proof:}

Let \(m := \inf_{[a,b]} f, \quad M := \sup_{[a,b]} f\) we know by the previous theorem that \(f\) is bounded, so \(m, M \in \Reals\).

By the Bolzano-Weierstrass theorem, there exist convergent subsequences \(\{x_{n_k}\}, \{x_{m_k}\} \subset [a,b]\) such that

\[
    \lim_{k \to \infty} f(x_{n_k}) = m, \quad \lim_{k \to \infty} f(x_{m_k}) = M
\]

Since \([a,b]\) is closed, the limits of these subsequences \(x_m, x_M \in [a,b]\).  Then by continuity of \(f\):

\[
    f(x_m) = \lim_{k \to \infty} f(x_{n_k}) = m, \quad f(x_M) = \lim_{k \to \infty} f(x_{m_k}) = M
\]

\QED

\subsection{Extreme Points}

If for set \(A \subseteq \Reals\) with \(f : A \to \Reals\) holds that 

\[
    f(x_m) = \inf_{x \in A} f(x) =: m \quad \text{or} \quad f(x_M) = \sup_{x \in A} f(x) =: M
\] 

then \(x_m\) is called a \emph{point of minimum} and \(x_M\) a \emph{point of maximum} of \(f\) on \(A\). Also, 
\(f(x_m)\) is called the \emph{minimum} and \(f(x_M)\) the \emph{maximum} of \(f\) on \(A\). These are the 
\emph{extreme values}.

\subsection{Middle Value Theorem}

If \(f: [a, b] \to \Reals\) is continuous and \(f(a) \ne f(b)\), then for every \(y_0\) between \(f(a)\) and \(f(b)\) 
there exists \(x_0 \in (a, b)\) such that \(f(x_0) = y_0\).

\subsection{Continuity of the Inverse by a decreasing Function}

Given a monotonic decreasing continuous function \(f\) on \([a, b]\) with \(f(a) = c\) and \(f(b) = d\), then the inverse function \(f^{-1}: [d, c] \to [a, b]\) is also
continuous and monotonic.

\textbf{Proof:}

\subsubsection{Continuity of the Root and Logarithm Function}

\(x \mapsto \sqrt[n]{x}\) is continuous on \([0, \infty)\)

\(x \mapsto \log_n{x}\) is continuous on \((0, \infty)\)

\subsection{Lipschitz Continuity}

A function \(f: D \subset \Reals \to \Reals\) is \emph{Lipschitz continuous} if:

\[
    \exists L > 0 \text{ such that } |f(x) - f(y)| \le L |x - y| \quad \forall x, y \in D
\]

\begin{itemize}

    \item Every Lipschitz continuous function is uniformly continuous.

    \item The smallest such \(L\) is called the \emph{Lipschitz constant}.

\end{itemize}

\subsection{Fixed Point}

A \emph{fixed point} of a function \(f\) is a point \(x^*\) such that:

\[
    f(x^*) = x^*
\]

\subsection{Fixed Point on Interval}

Given a continuous function \(f:[a,b] \to \Reals\) with \(f([a,b]) \subseteq [a,b]\), then there exists at least one fixed 
point \(x^* \in [a,b]\) such that \(f(x^*) = x^*\).

\subsection{Banach Fixed Point Theorem}

Let \((X, d)\) be a complete metric space, and \(f: X \to X\) a \emph{contraction mapping}, i.e.,

\[
    \exists L < 1 \text{ such that } d(f(x), f(y)) \le L \cdot d(x, y)
\]

Then:

\begin{itemize}

    \item \(f\) has a unique fixed point \(x^* \in X\), such that \(f(x^*) = x^*\)

    \item Iterating \(x_{n+1} = f(x_n)\) converges to \(x^*\)

\end{itemize}

\emph{Interpretation and Meaning}

The Banach Fixed Point Theorem ensures:

\begin{itemize}

    \item Existence and uniqueness of solutions (fixed points)

    \item Convergence of approximation by iteration

    \item Powerful tool in numerical methods and differential equations

\end{itemize}

\subsubsection{A Priori and A Posteriori Approximations}

\emph{A priori estimate:}

\[
    |x_n - x^*| \le \frac{L^n}{1 - L} |x_1 - x_0| < \epsilon
\]

\emph{A posteriori estimate:}

\[
    |x_n - x^*| \le \frac{L}{1 - L} |x_n - x_{n-1}| \epsilon
\]

\subsubsection{Steps for a Fixed Point Exercise}

Given \(f: [a, b] \to [a, b]\), to prove existence and convergence:

\begin{enumerate}

    \item Show monotonicity: \(f\) is increasing or decreasing.

    \item Check interval preservation: \(f([a, b]) \subseteq [a, b]\)

    \item Check Lipschitz continuity: Find \(L < 1\)

    \item Fixed point iteration: Choose \(x_0\), compute \(x_{n+1} = f(x_n)\)

    \item Apply a priori or a posteriori bound

\end{enumerate}

\textbf{Example:} 

\(f(x) = \frac{1}{x} + 3\) on \([2, 5]\)

\textbf{Step 1: Monotonicity}

In this case we will assume that it is monotonous. 

\textbf{Step 2: Domain check} 
 
 \(x \in [2, 5] \Rightarrow \frac{1}{x} \in [0.2, 0.5] \Rightarrow f(x) \in [3.2, 3.5] \subset [2, 5]\)

 \textbf{Step 3: Lipschitz constant}

\[
    f'(x) = -\frac{1}{x^2} \Rightarrow |f'(x)| \le \frac{1}{2^2} = \frac{1}{4} < 1
    \Rightarrow \text{Lipschitz with } L = \frac{1}{4} < 1
\]


\textbf{Step 4: Fixed point iteration}

\[
    x_0 = 3, \quad x_1 = f(3) = \frac{1}{3} + 3 = \frac{10}{3}, \quad x_2 = f(x_1), \dots
\]

\textbf{Step 5: Approximations}

\emph{A priori}

\[
    |x_n - x^*| \le \frac{L^n}{1 - L} |x_1 - x_0|
\]

\[
    |x_n - x^*| \le \frac{\frac{1}{4}^n}{1 - \frac{1}{4}} |\frac{10}{3}  - 3|
\]

\begin{align*}
    |x_n - x^*| &\le \frac{\left(\frac{1}{4}\right)^n}{1 - \frac{1}{4}} \cdot \left|\frac{10}{3} - 3\right| \\
    &= \frac{\left(\frac{1}{4}\right)^n}{\frac{3}{4}} \cdot \frac{1}{3}\\
    &= \frac{4}{3} \cdot \left(\frac{1}{4}\right)^n \cdot \frac{1}{3}\\
    &= \frac{4}{9} \cdot \left(\frac{1}{4}\right)^n
\end{align*}

For \(\varepsilon = 0,1\)

\begin{align*}
    \frac{4}{9} \cdot \left(\frac{1}{4}\right)^n &< \varepsilon \\
    \left(\frac{1}{4}\right)^n &< \frac{9}{4} \varepsilon \\
    \ln\left(\left(\frac{1}{4}\right)^n\right) &< \ln\left(\frac{9}{4} \varepsilon\right) \\
    n \cdot \ln\left(\frac{1}{4}\right) &< \ln\left(\frac{9}{4} \varepsilon\right) \\
    n &> \frac{\ln\left(\frac{9}{4} \varepsilon\right)}{\ln\left(\frac{1}{4}\right)} \\
    n &> \frac{\ln\left(\frac{9}{4} \varepsilon\right)}{-\ln 4}
\end{align*}


\emph{A posteriori}

\[
    |x_n - x^*| \le \frac{L}{1 - L} |x_n - x_{n-1}| = \frac{\frac{1}{4}}{1 - \frac{1}{4}} |x_n - x_{n-1}| = \frac{1}{3} |x_n - x_{n-1}|
\]

\subsection{Uniform Continuity}

A function \(f: \mathcal{D} \subset \Reals \to \Reals\) is \emph{uniformly continuous} if:

\[
    \forall \varepsilon > 0, \exists \delta > 0 \text{ such that } |x - y| < \delta \Rightarrow |f(x) - f(y)| < \varepsilon \quad \forall x, y \in \mathcal{D}
\]

\textbf{Example:}

\(f(x) = \frac{1}{x}\) is not uniformly continuous on \((0, 1)\) 

Let \(\varepsilon = 1\). We want to find numbers in \((0,1)\) for \(x,y\) such that \(|x - y| < \delta\) but \(|f(x) - f(y)| \ge 1\). 

Choose \(x \in (0,1)\) with \(x < \delta\) and let \(y = \frac{x}{2}\). Then 

\[
    |x - y| = ||x - \frac{x}{2}| = \frac{x}{2} < \frac{\delta}{2} < \delta
\]

But, 

\[
    |f(x) - f(y)| = \left|\frac{1}{x} - \frac{1}{y}\right| = \left|\frac{1}{x} - \frac{2}{x}\right| = \left|-\frac{1}{x}\right| = \frac{1}{x} > 1
\]

Since \(\frac{1}{x} > 1\) for \(x < 1\), we have a contraction and therefore \(f\) is not uniformly continuous on \((0,1)\).

\subsection{Boundedness by Uniform Continuity}

If a function \(f: (a, b) \to \Reals\) is uniformly continuous on the open interval \((a, b)\), then \(f\) is bounded.

\subsection{Cauchy Sequence by Uniform Continuity}

Given \(\mathcal{D} \subseteq \Reals\) and a uniformly continuous function \(f: \mathcal{D} \to \Reals\).
Then for every Cauchy sequence \(\{x_n\} \subseteq \mathcal{D}\), the sequence \(\{f(x_n)\}\) is also a Cauchy sequence.
