\newpage
\section{Statistics}

\emph{Statistics} focuses on getting information not from probabilities, random variables and 
distributions, but from samples of data collected independently.

\subsection{Basic Nomenclature}

\begin{itemize}

    \item \emph{Population}: Set of all objects about which we make a statement.

    \item \emph{Sample}: Finite set of the Population. 

    \item \emph{Sample Perimeter}: Number of the samples. Like the number of participants. 

    \item \emph{Trait}: Special characteristic we are looking for. They can be discrete, continuous, non-numerical (can be converted to numerical).

    \item \emph{Trait-quantifier}: Amount of correspondence to a trait.
     
\end{itemize}

\subsection{Frequency}

Given a trait \(X\) with some quantifiers \((a_i)_{i \in I}\) for a discrete set \(I\). Given a 
sample of observations \(x_1, x_2, \dots, x_n\) of \(X\). The \emph{absolute frequency} \(h_i\) of \(a_i\) is 
defined as 

\[
    h_i = \sum_{j = 1}^{n} \text{ind}_{x_j = a_i} i \in I
\]

The \emph{relative frequency} is defined as  

\[
    r_i = \frac{h_i}{n}, i \in I
\]


