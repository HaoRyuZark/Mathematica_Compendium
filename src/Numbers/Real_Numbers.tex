\newpage
\section{Real Numbers}

A way of defining the \emph{real numbers} is via the \emph{Cauchy sequences}. Another way is via 
the axioms of a field with the order axioms and the \emph{completeness axiom}.

\begin{itemize}

    \item \emph{Associativity:} \((a \ast b) \ast c = a \ast (b \ast c)\) for all \(a,b,c \in S\)
    
    \item \emph{Commutativity:} \(a \ast b = b \ast a\) for all \(a,b \in S\)
    
    \item \emph{Identity Element:} There exists \(e \in S\) such that \(a \ast e = e \ast a = a\) for all \(a \in S\)
    
    \item \emph{Inverse Element:} For every \(a \in S\), there exists \(a^{-1} \in S\) such that \(a \ast a^{-1} = a^{-1} \ast a = e\)
    
    \item \emph{Distributivity:} \(a \circ (b \bullet c) = (a \circ b) \bullet (a \circ c)\) and/or \((b \bullet c) \circ a = (b \circ a) \bullet (c \circ a)\)

\end{itemize}

These rules apply for the addition and multiplication for real numbers.

We also refresh the order axioms.

\begin{enumerate}[label=\Roman*.]
    
	\item \emph{Trichotomy:} Exactly one of the following holds:
 
		  \[
 	   			a < b, \quad a = b, \quad a > b
    	   \]

    \item \emph{Transitivity:} If \( a < b \) and \( b < c \), then \( a < c \).

    \item \emph{Additive Compatibility:} If \( a < b \), then \( a + c < b + c \).

    \item \emph{Multiplicative Compatibility:} If \( 0 < a \) and \( 0 < b \), then \( 0 < ab \).

\end{enumerate}

\subsection{Properties derivated from the Axioms}

Some proofs of these properties can be found in the section about the algebraic structures.

\begin{itemize}

	\item The neutral and inverse elements of the addition and multiplication are well defined.

	\item \(a + b = c\) and \(a + c = 0 \implies b = c\)

	\item \(a \ne 0\), \(ab = 1\) and \(ac = 1 \implies b = c\)

	\item \(a \cdot 0 = 0\) 

	\item \(-a = (-1)a\)
	
	\item \(-(-a) = a\)

	\item \((-1)^2 = 1\)
		
	\item \(\left(a^{-1}\right)^{-1} = a, \forall \in \Reals, a \ne 0\)

\end{itemize}

\subsection{Division Algorithm}

For \(a, b \in \Integers, a \ge b, b \ne 0 \exists q,r \in Integers : a = bq + r\) with 
\(0 \le r < q\).

\textbf{Proof:}

For \(a = b, q = 1, r = 0\) which are well-defined by the axioms. For \(a > b\) we have 
that if \(a\) is a multiple of \(b\) then \(r = 0\). Lastly, for the last case, let us assume there exists 
\(bq + r = a\) and also there exist \(b'q + r' = a\) then 

\begin{align*}
	bq + r &= b'q + r' \\ 
	r &= b'q + r' - bq
\end{align*}

Then, \(0 \le r < q \implies  -q < r < q \), thus 

\[
	-q < b'q + r' - bq < q  \implies b'q - bq = 0
\]

Thus, the theorem is proven. 

\QED 

\subsection{Existence of Solutions}

Given \(a,b \in \Reals\), then 

\[
	!\exists x \in \Reals, \text{ such that }, a + x = b
\]

Also, if \(a \ne 0\) 

\[
	!\exists x \in \Reals, \text{ such that }, ax = b
\]

\textbf{Proof:}

For the first proposition, let \(x = -a + b\) 

\begin{align*}
	a + x &= a -a + b \\
		 &= b
\end{align*}

For the second, let \(x = a^{-1}b\)

\begin{align*}
	ax &=  a a^{-1} b \\ 
	   &= 1 \cdot b  
\end{align*}

\QED

\subsection{The Completeness Axiom}

All non-empty upper-bounded subsets of the real numbers have a \emph{supremum} \(s \in \Reals\).

\subsection{Minima in the Real Numbers}

As a consequence of the last axiom each non-empty lower-bounded subset of real numbers has a minimum. 

\textbf{Proof:}

The main idea is that the supremum times minus 1 of a set is the minimum of another set. 

We define \(- A := \{-a \mid a \in A\}\) then \(-A \ne \emptyset\) and is upper-bounded. 
By definition \(s\) is less or greater than all the element of the set \(A\).

\begin{align*}
	s &\le a, \forall a \in A	\\ 
	-s &\ge -a, \forall a \in A
\end{align*}

Thus, \(-s = \text{sup}-A\). 

Now we want to prove that \(-\text{sup}-A\) is first, the lower bound of \(A\) and second, it is the greatest upper bound.

For the first part 

\begin{align*}
	\text{sup}-A &\ge -a \forall a \in A \\ 
	- \text{sup} -A &\le a \forall a \in A
\end{align*}

Thus, \(-\text{sup}-A\) is in fact a lower bound of \(A\). For the second part let us 
assume that there exists some element \(s\) such that \(s > -\text{sup}-A\) and \(s \le a, \forall a \in A\), then 

\[
	-s < \sup -A  \text{ and } -s \ge -a, \forall a \in A
\]

This is a contradiction because we have an upper bound, which is not an upper bound. Therefore, 
\(-\text{sup}-A\) is the infimum of \(A\).

\QED

\subsection{Approximation of the Supremum}

Given a non-empty, upper-bounded set \(A \subseteq \Reals\) and  \(s = \text{sup}A\). Then there is for each \(y < s\) 
an \(x \in A\), such that 

\[
	y < x \le s
\]

An analogous theorem can be written for the minimum as well.

\textbf{Proof:}

If \(x \in A\) then the proof is completed. For the case \(s \notin A\), if there existed some 
\(y^* \in \Reals\) where \(y^* < s\) and \(y^* \ge x\) for all \(x \in A\) then we sill have a contradiction to 
the assumption that \(s\) is the supremum of \(A\).

\QED

\subsection{Supremum of the Naturals}

\(\Naturals\) is not upper-bounded.

\textbf{Proof:}

Assume \(a = \text{sup}\Naturals\). Since \(\Naturals \subset \Reals\) is not empty it has an upper-bound by 
the completeness axiom. 

By the definition of the supremum \(a - 1 \in \Naturals\) is not an upper-bound of \(\Naturals\), therefore 
\(a - 1 < k, k \in \Naturals\). 

Then, the following also applies

\[
	a < k + 1
\]

This is a contradiction to the assumption that \(a\) our supremum, thus our assumption has been 
proven false.

\subsection{Remainder of Archimedes and Eudoxus}

\emph{Archimedes}

For all pair of positive real numbers \(x,y\) exists some \(n \in \Naturals\) such that: 

\[
	nx > y
\]

\emph{Eudoxus}

For all positive real numbers there exists an \(n \in \Naturals\), such that 

\[
	\frac{1}{n} < x
\]

\textbf{Proof Eudoxus:}

\(x \in \Reals_{> 0} \implies \frac{1}{x} \in \Reals_{> 0}\). By the Archimedian property, \(\exists n \in \Integers_{0 >}, \frac{1}{x} < n\). 
Equivalently \(\frac{1}{n} < x\). 

\QED 

\subsection{Density of the Rationals in  the Reals}

For each pair \(a, b \in \Reals\) there exists some \(q \in \Rationals\) such that 

\[
	a < q < b
\]

\textbf{Proof:}

Let \(q = \frac{m}{n}, m \in \Integers, n \in \Naturals\) to prove \(a < \frac{m}{n} < b\) equivalently \(an < m < bn\).

For \(b - a > 0\) we know \(\frac{1}{n} < b - a\) or \(a < b - \frac{1}{n}\), by the Archimedian property. 

Next take the smallest \(m \in \Integers\) such that \(m - 1 \le na < m\) or \(a < \frac{m}{n}\). This gives us the first half of 
our theorem.

Finally, for the second half 

\begin{align*}
	m - 1 &\le na \\ 
		&= na + 1 \\ 
		&< n(b - \frac{1}{n}) + 1 \text{ using our result from the Archimedian property }\\ 
		&= nb
\end{align*}

This tells us that \(m < nb \implies \frac{m}{n} < b\).

\QED

\subsection{Existence of Square Roots}

For each \(x \in [0, \infty], !\exists  r \in [0, \infty) \text{ s. d. } r^2  =x\). Also,
\(\sqrt{x}:= r\) and we call \(x \mapsto \sqrt{x}\) the \emph{root function}.

\textbf{Proof:}

Let \(T = \{x \in [0, \infty) \mid x < y\}\) and let us define \(a = y^2\) as the supremum of \(T\). 

To prove that the equality holds we will prove that the other order cases can not be true. 

Assume \(a < y\), because \(a - y < 0\), we know that \(\frac{1}{n} < a - y\) and \( a + \frac{1}{n} < y\). Thus, \(a + \frac{1}{n} \in T\), but 
because \(a\) is the supremum this contradiction because our supremum is not the smallest upper-bound. 

For the second case, assume \(a > y\), by applying the same reasoning with \(a - \frac{1}{n}\), we will arrive at a similar contradiction 

Assume \(a > y\), then \(a - y > 0 \implies \frac{1}{n} < a - y, \implies a - \frac{1}{n} < y\)

But then \(a  - \frac{1}{n}\) is an upper bound and this contradicts our assumption.
Thus, by elimination \(a^2 = y\)

\QED

\subsection{Irrational Numbers}

\[
	\Reals \setminus \Rationals \text{ or } \Rationals^C := \{x \in \Reals \mid x \notin \Rationals\}
\]

\subsection{Absolute Value for Real Numbers}

\[
	|x| := \sqrt{x*x}
\]

\subsection{Norms Remainder}

Properties of a norm 

\begin{itemize}
	
	\item \(|x| \ge 0, \forall x \in \Reals\)
	
	\item \(|x| = 0, \implies x = 0\)

	\item \(|\alpha x| = |\alpha| |x|, \forall x, \alpha \in \Reals\)

	\item \(|x + y| \le |x| + |y| \forall x,y \in \Reals\)

\end{itemize}

\subsection{Distance}

We define the \emph{distance} between two real numbers as 

\[
	d(x,y) = | x - y |
\]

With the following properties

\begin{itemize}

	\item \(d(x,y) \ge 0, \forall x,y \in \Reals\)
	
	\item \(d(x,y) = 0, \iff x = y \)

	\item \(d(x,y) = d(y,x)\)
	
	\item \(d(x,y) \ge d(x,y) + d(z,y), \forall x,y \in \Reals\)

\end{itemize}

\subsection{Nested Interval Property}

For each \(n \in \Naturals\), assume we are given a closed interval  \(I_n = [a, b] = \{x \in \Reals \mid a_n \le x \le b_n\}\). 
Also assume each \(I_n\) contains \(I_{n + 1}\). Then the nested sequence of closed intervals \(I_1 \supseteq I_2 \supseteq I_3 \supseteq \ldots\)
has a non-empty intersection. 

\textbf{Proof:}

By definition every \(I_n\) is non-empty, thus the sequences \((a_n)_{n \in \Naturals}\) and \((b_n)_{n \in \Naturals}\) are well-defined. 
Also, every set \(A = \{a_n: n \in \Naturals\}\) is bounded above by the corresponding \(b_n\), and this is guaranteed by 
the axiom of completeness.

Let \(x = \sup A\). Then consider \(I_n = [a_n, b_n]\) 

\[
	a_n \le x \le b_n, \forall n \in \Naturals \implies x \in I_n
\]

Thus, \(x \in \bigcap_{n = 1}^{\infty} I_n\) and the intersection is non-empty. 

\QED

\subsection{Definition as Cauchy Sequences}
 
Let \( K \) be an ordered field. On the set
		
\[
	\operatorname{ch}(K) := \{ x : \Naturals \to K \mid x \text{ is a Cauchy sequence} \}
\]
		
and on the set
	
\[
	c(K) := \{ x : \Naturals \to K \mid x \text{ is a convergent sequence} \}
\]

we can define an addition and a multiplication as follows:

If \( x = {(x_n)}_{n \in \Naturals} \) and \( y = {(y_n)}_{n \in \Naturals} \) are Cauchy sequences 
(respectively, convergent sequences), then their sum is defined as
	
\[
	x + y := {(x_n)}_{n \in \Naturals} + {(y_n)}_{n \in \Naturals} := {(x_n + y_n)}_{n \in \Naturals}
\]
	
and their product is defined as

\[
	x \cdot y := {(x_n)}_{n \in \Naturals} \cdot {(y_n)}_{n \in \Naturals} := {(x_n \cdot y_n)}_{n \in 
	\Naturals}
\]

The sum and product satisfy all field axioms except for the existence of the multiplicative inverse.
The zero element is \( 0_{\Naturals} = (0, 0, \ldots) \), the unit element is \( 1_{\Naturals} = (1, 1, 
\ldots) \), and the additive inverse of \( x = {(x_n)}_{n \in \Naturals} \) is 
\( -x = {(-x_n)}_{n \in \Naturals} \). We demonstrate the distributive law as an example:

Let \( x = {(x_n)}_{n \in \Naturals}, y = {(y_n)}_{n \in \Naturals}, z = {(z_n)}_{n \in \Naturals} \) be 
Cauchy sequences (convergent sequences). Then we have:

\begin{align*}
	x(y + z) &= {(x_n)}_{n \in \Naturals} \cdot \left( {(y_n)}_{n \in \Naturals} + {(z_n)}_{n \in \Naturals} \right)\\
	&= {(x_n)}_{n \in \Naturals} \cdot {(y_n + z_n)}_{n \in \Naturals}\\ 
	&= {(x_n (y_n + z_n))}_{n \in \Naturals}\\
	&= {(x_n y_n + x_n z_n)}_{n \in \Naturals}\\
	&= {(x_n y_n)}_{n \in \Naturals} + {(x_n z_n)}_{n \in \Naturals}\\
	&= xy + xz.
\end{align*}

We now aim to construct the ordered field \( \Reals \) of the real numbers;
it will have the following properties:

\begin{itemize}
	
	\item There exists an injective mapping \( j : \Rationals \to \Reals \) which respects addition, 
    	  multiplication, and order, such that the following holds:
		
		For all \( z, w \in \Reals \) with \( z < w \), there exists an \( x \in \Rationals \) such that
		
		\[
			z < j(x) < w
		\]

	\item Every Cauchy sequence in \( \Reals \) converges.

\end{itemize}

Via \(j\), we identify \( \Rationals \) with \( j(\Rationals) \) and consider \( \Rationals \) as a 
subset of \( \Reals \). In \( \Reals \), the following will additionally hold:

\begin{itemize}

	\item For all \( y > 0 \) and \( n \in \Naturals \), the equation \( x^n = y \) has a solution.

	\item Every bounded above subset of \( \Reals \) has a supremum.

\end{itemize}

We define the following relation on the set \( \operatorname{ch}(\Rationals) \) of all Cauchy sequences 
in \( \Rationals \):

\[
	x \sim y \quad \text{if and only if} \quad x - y \text{ is a null sequence}
\]

That is, \( {(x_n)}_{n \in \Naturals} \sim {(y_n)}_{n \in \Naturals} \) if and only if

\[
	x_n - y_n \to 0 \quad (n \to \infty)
\]

\subsection{Definition}

The set

\[
	\Reals := \{ {[x]}_{\sim} : x \in \operatorname{ch}(\Rationals) \}
\]

is called the set of real numbers.

Analogous to the construction of the rational numbers, the real numbers consist of equivalence classes.
Roughly speaking, an equivalence class consists of those Cauchy sequences in \( \Rationals \) that exhibit 
the same limit behavior.

Equipped with the addition

\[
	+ : \Reals \times \Reals \to \Reals, \quad [x], [y] \mapsto [x] + [y] := [x + y]
\]

and the multiplication

\[
	\cdot : \Reals \times \Reals \to \Reals, \quad [x], [y] \mapsto [x] \cdot [y] := [xy]
\]

\( \Reals \) is a field. The zero element is \( [0_{\Naturals}] \), and the unit element is 
\( [1_{\Naturals}] \).

