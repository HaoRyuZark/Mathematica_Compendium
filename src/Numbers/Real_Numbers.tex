\newpage
\section{Real Numbers}

A way of defining the \emph{real numbers} is via the \emph{Cauchy sequences}. Another way is via 
the axioms of a field. 

\subsection{Definition as Cauchy Sequences}
 
Let \( K \) be an ordered field. On the set
		
\[
	\operatorname{ch}(K) := \{ x : \Naturals \to K \mid x \text{ is a Cauchy sequence} \}
\]
		
and on the set
	
\[
	c(K) := \{ x : \Naturals \to K \mid x \text{ is a convergent sequence} \}
\]

we can define an addition and a multiplication as follows:

If \( x = {(x_n)}_{n \in \Naturals} \) and \( y = {(y_n)}_{n \in \Naturals} \) are Cauchy sequences 
(respectively, convergent sequences), then their sum is defined as
	
\[
	x + y := {(x_n)}_{n \in \Naturals} + {(y_n)}_{n \in \Naturals} := {(x_n + y_n)}_{n \in \Naturals}
\]
	
and their product is defined as

\[
	x \cdot y := {(x_n)}_{n \in \Naturals} \cdot {(y_n)}_{n \in \Naturals} := {(x_n \cdot y_n)}_{n \in 
	\Naturals}
\]

The sum and product satisfy all field axioms except for the existence of the multiplicative inverse.
The zero element is \( 0_{\Naturals} = (0, 0, \ldots) \), the unit element is \( 1_{\Naturals} = (1, 1, 
\ldots) \), and the additive inverse of \( x = {(x_n)}_{n \in \Naturals} \) is 
\( -x = {(-x_n)}_{n \in \Naturals} \). We demonstrate the distributive law as an example:

Let \( x = {(x_n)}_{n \in \Naturals}, y = {(y_n)}_{n \in \Naturals}, z = {(z_n)}_{n \in \Naturals} \) be 
Cauchy sequences (convergent sequences). Then we have:

\begin{align*}
	x(y + z) &= {(x_n)}_{n \in \Naturals} \cdot \left( {(y_n)}_{n \in \Naturals} + {(z_n)}_{n \in \Naturals} \right)\\
	&= {(x_n)}_{n \in \Naturals} \cdot {(y_n + z_n)}_{n \in \Naturals}\\ 
	&= {(x_n (y_n + z_n))}_{n \in \Naturals}\\
	&= {(x_n y_n + x_n z_n)}_{n \in \Naturals}\\
	&= {(x_n y_n)}_{n \in \Naturals} + {(x_n z_n)}_{n \in \Naturals}\\
	&= xy + xz.
\end{align*}

We now aim to construct the ordered field \( \Reals \) of the real numbers;
it will have the following properties:

\begin{itemize}
	
	\item There exists an injective mapping \( j : \Rationals \to \Reals \) which respects addition, 
    	  multiplication, and order, such that the following holds:
		
		For all \( z, w \in \Reals \) with \( z < w \), there exists an \( x \in \Rationals \) such that
		
		\[
			z < j(x) < w
		\]

	\item Every Cauchy sequence in \( \Reals \) converges.

\end{itemize}

Via \(j\), we identify \( \Rationals \) with \( j(\Rationals) \) and consider \( \Rationals \) as a 
subset of \( \Reals \). In \( \Reals \), the following will additionally hold:

\begin{itemize}

	\item For all \( y > 0 \) and \( n \in \Naturals \), the equation \( x^n = y \) has a solution.

	\item Every bounded above subset of \( \Reals \) has a supremum.

\end{itemize}

We define the following relation on the set \( \operatorname{ch}(\Rationals) \) of all Cauchy sequences 
in \( \Rationals \):

\[
	x \sim y \quad \text{if and only if} \quad x - y \text{ is a null sequence}
\]

That is, \( {(x_n)}_{n \in \Naturals} \sim {(y_n)}_{n \in \Naturals} \) if and only if

\[
	x_n - y_n \to 0 \quad (n \to \infty)
\]

\subsection{Definition}

The set

\[
	\Reals := \{ {[x]}_{\sim} : x \in \operatorname{ch}(\Rationals) \}
\]

is called the set of real numbers.

Analogous to the construction of the rational numbers, the real numbers consist of equivalence classes.
Roughly speaking, an equivalence class consists of those Cauchy sequences in \( \Rationals \) that exhibit 
the same limit behavior.

Equipped with the addition

\[
	+ : \Reals \times \Reals \to \Reals, \quad [x], [y] \mapsto [x] + [y] := [x + y]
\]

and the multiplication

\[
	\cdot : \Reals \times \Reals \to \Reals, \quad [x], [y] \mapsto [x] \cdot [y] := [xy]
\]

\( \Reals \) is a field. The zero element is \( [0_{\Naturals}] \), and the unit element is 
\( [1_{\Naturals}] \).

