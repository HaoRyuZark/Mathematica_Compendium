\newpage
\section{Order}

\subsection{Principle of the Ordering}

Every non-empty set of natural numbers has a minimum.

\subsection{Order in Fields}

An \emph{ordered field} is a field \(F\) equipped with a total order \( < \) that is compatible with 
the field operations + and \( \cdot \). That is, the order satisfies both algebraic and ordering 
properties.

\subsubsection{Order Axioms for Fields}

A field \(F\) is called an \emph{ordered field} if it satisfies the following properties for all 
\( a, b, c \in F \):

\begin{enumerate}[label=\Roman*.]
    
	\item \emph{Trichotomy:} Exactly one of the following holds:
 
		  \[
 	   			a < b, \quad a = b, \quad a > b
    	   \]

    \item \emph{Transitivity:} If \( a < b \) and \( b < c \), then \( a < c \).

    \item \emph{Additive Compatibility:} If \( a < b \), then \( a + c < b + c \).

    \item \emph{Multiplicative Compatibility:} If \( 0 < a \) and \( 0 < b \), then \( 0 < ab \).

\end{enumerate}

These properties ensure that arithmetic operations respect the order structure.

\subsubsection{Consequences of the Order Axioms}

\begin{itemize}

	\item \( a < b \implies -b < -a \)

	\item \( a < b \land c < 0 \implies ac > bc \)

    \item \(a < 0, b < 0 \implies a + b < 0\) the contrary applies to the multiplication \(ab > 0\)

    \item \(a > 0, b > 0 \implies a + b > 0\)
	
    \item Squares are always non-negative: \( a^2 \ge 0 \)

	\item The order is total: any two elements are comparable

\end{itemize}

\subsubsection{Examples of Ordered Fields}

\begin{itemize}

	\item \( \Rationals \): Rational numbers with the usual order

	\item \( \Reals \): Real numbers with the usual order

	\item \( \Complex \): Complex numbers are not an ordered field, since \( i^2 = -1 < 0 \) 
	      would violate positivity of squares

\end{itemize}

