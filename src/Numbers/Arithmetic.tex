\newpage
\section{Arithmetic}

\subsection{Even and Odd numbers}

A natural number \(a\) is \emph{even} if \(a = 2k\) with \(k \in \Naturals\).

A natural number \(b\) is \emph{odd} if \(b = 2q - 1\) with \(q \in \Naturals\).

\subsection{Perfect Squares}

A natural number \(a\) has a \emph{perfect square} if \(\sqrt{a} \in \Naturals\).

\subsection{Division Algorithm}

Given any two integers \(a\) and \(b\) with \(a > 0\), there exists unique integers \(q\) and \(r\) such that 
\(b  =qa + r\) with \(0 \ge r < a\).

\textbf{Proof:}

Let \(S = \{b - xa \mid b - xa \ge 0, x \in \Integers\}\)

By the well ordering principle of the natural numbers we know that our set has 
minimum \(r\). 

First we prove \( 0 \le r\)

If \(b \ge 0\) then let \(x = 0 \implies b \in S\).
For \(b < 0\) then let \(x = b\). Thus 

\[
	b - ba  = b(1 - a) \implies b - ba \in S
\]

Now to prove that \(r < a\), assume \( r \ge a\), but then 


\begin{align*}
	b  - qa &\ge a \\
	b  - qa - a &\ge 0\\ 
	b  - a(q + 1) &\ge 0\\ 
	b  - a(q + 1) &\ge 0 \implies b - a(q + 1) \in S\\ 
\end{align*}

And \(b - a(q + 1) < b - aq = r \implies r < a\) therefore, we have a contradiction.

Now we will prove the uniqueness. 

Assume \(pq + r = a = pq' + r'\) with \(q \ne q' \) and \(r \ne r'\)

\[	
	r - r' = a(q - q') = 0
\]

Using the property of inverses of \(\Integers \setminus \{0\}\) as a Field we know that 
\(r = r'\) and \(q - q'\).

\QED 

\subsection{Prime Numbers}

A \emph{prime number} \(p \in \Naturals \setminus \{1\}\) fulfills the following condition:

\begin{enumerate}[label= \Roman*.]
	\item \(p\) is only divisible by itself and 1.
\end{enumerate}

\subsection{Pythagorean Triplets}

\emph{Pythagorean triplets} are 3 natural numbers such that 

\[
	a^2 + b^2 = c^c
\]

\subsection{Divisibility}

A number \(a \in \Integers\) is divisible a number \(b \in \Integers\) if 

\[
	a = kb \quad k \in \Integers
\]

\subsection{Division}

Given \(a,b \in \Integers\). There exists unique \(q,r \in \Integers\) such that 

\[
	a = qb + r \quad 0 \le r < |b|
\]

\subsection{Greatest Common Denominator}

For two integers \(a, b \in \Integers\) with \((a,b) \neq (0,0)\), the positive
integer \(t\) satisfying

\begin{itemize}

	\item \(t \mid a\) and \(t \mid b\),

	\item for all \(c \in \Integers\) with \(c \mid a\) and \(c \mid b\), it follows that \(c \mid t\),

\end{itemize}

is called the \emph{greatest common divisor} (gcd) of \(a\) and \(b\) denoted as \(t = \gcd(a,b)\).

If \(t = 1\), then \(a\) and \(b\) are said to be \emph{coprime} or \emph{relatively prime}.

\subsection{The Euclidean Algorithm}

If \(a = bq + r\) then \(\gcd(a, b) = \gcd(b, r)\). 

\textbf{Input:} \(a, b \in \mathbb{Z}\), \(a \neq 0 \neq b\).  
\textbf{Output:} The greatest common divisor \(t\) of \(a\) and \(b\).

\textbf{Algorithm:}

\begin{enumerate}

	\item If \(a = 0\), then return \(|b|\). If \(b = 0\), return \(|a|\).

	\item Set \(r_1 := a\); \(r_2 := b\); \(k := 2\).

	\item As long as \(r_k \neq 0\), define \(r_{k+1} := r_{k-1} \bmod r_k\).

	\item As soon as \(r_k = 0\), return \(r_{k-1}\).

\end{enumerate}

In simpler terms, as long the remainder is not zero, we keep dividing the previous remainder by the current
remainder and updating the remainders. When we reach a remainder of zero, the previous remainder is the gcd. 

\textbf{Proof:}

Notice, that the algorithm terminates because the sequence given \(r_n\) is monotone decreasing due to \(0 \le r < b\) and 
the well ordering principle of the natural numbers.

For the correctness: given is \(a = bq + r\) to show that \(\gcd(a, b) = \gcd(b, r)\). This is like saying 

\begin{align*}
	\gcd(a,b) & = gcd(b, r) \\ 
	\gcd(b,r) & = \gcd(r, r_1)\\
	\gcd(r, r_1) & = \gcd(r_1, r_2) \\
	& \vdots \\
	\gcd(r_{n-1}, r_{n}) & = \gcd(r_n, 0) = r_n 
\end{align*}

Let \(d\) be a common divisor of \(a\), \(b\). Then \(d \mid a\) and \(d \mid b\), so \(d \mid ((a - bq) = r) \implies d \mid r \).

Let \(e\) be a common divisor of \(b\), \(r\). Then \(e \mid b\) and \(e \mid r\), so \(e \mid ((bq + r) = a) \implies e \mid a \).

Thus, the set of common divisors of \(a\), \(b\) is the same as the set of common divisors of \(b\), \(r\),
so their greatest common divisors are equal.

\QED 

\textbf{Example:}

Find \(\gcd(252, 198)\):

\begin{align*}
	r_{n - 1} & = q_{n - 1} r_n + r_{n + 1} \\
	252 & = 1 \cdot 198 + 54  \\
	198 & = 3 \cdot 54 + 36   \\
	54  & = 1 \cdot 36 + 18   \\
	36  & = 2 \cdot 18 + 0
\end{align*}

So, \(\gcd(252, 198) = 18\). 

\subsection{Euclidean Algorithm as Linear Map}

Given \(E := \{(a,b)^T \mid a, b \in \Integers, b \ne 0 \}\) and 

\[
	\varepsilon: E \to \Integers^{2 \times 1} : 
	\begin{bmatrix}
		a \\ 
		b
	\end{bmatrix}
	\mapsto
	\begin{bmatrix}
		b \\
		a \mod b
	\end{bmatrix}
\]

Then the Euclidean algorithm can be described as repeated application of \(\varepsilon\) until the second
component is zero. The first component at that point is the gcd.

\[
	\varepsilon^{k - 1}
	\begin{bmatrix}
		a \\ 
		b
	\end{bmatrix}
	= \varepsilon \circ \cdots \circ \varepsilon
	\begin{bmatrix}
		t \\ 
		0
	\end{bmatrix}
\]

\(t = \gcd(a,b)\).

Also, we can express \(\varepsilon\) as a matrix multiplication:

\[
	\varepsilon
	\begin{bmatrix}
		a \\ 
		b
	\end{bmatrix}
	=
	\begin{bmatrix}
		0 & 1 \\
		1 & -q
	\end{bmatrix}
	\begin{bmatrix}
		a \\ 
		b
	\end{bmatrix}
	= 
	\begin{bmatrix}
		b \\ 
		r
	\end{bmatrix}
\]

with \(a = qb + r\) and \(q \in \Integers, 0 \le q < |b|\).

\subsection{Bezout Identity}

This is one of the most important properties of the \emph{gcd}. 
Given \(a, b \in \Integers\), not both zero, there exist integers \(\alpha, \beta \in \Integers\) such that

\[
	\gcd(a, b) = a\alpha + b\beta
\]

This equation comes directly from \emph{Euclid's Algorithm}. For the intuition, when writing your iterations 
you can manipulate them to arrive at the solution.

\textbf{Proof:}

Given 

\[
	M := \{a\alpha + b \beta \mid a\alpha + b \beta > 0 \land \alpha \in \Integers \land \beta \in \Integers\}
\]

\(M\) is not empty because of \(\alpha, \beta \in M\), and it has a minimum by the well ordering principle 
of the natural numbers which we will call \(c\).

Then, let \(k \in M\). Then there exist \(q \in \Naturals_0\) and \(r \in \Naturals_0\) with \(r < c\) such 
that

\[
	k = qc + r
\]

because \(k\) and \(c\) are elements of \(M\) we should be able to find \(r_1, r_2, s_1, s_2\)  such that 

\[
	k = r_1 a + s_1 b \quad c = r_2 a + s_2 b
\]

Now let us

\begin{align*}	
	k = r_1 a + s_1 b &= qc + r \\ 
		r_1 a + s_1 b &= q(r_2 a + s_2 b) + r \\ 
		r &= r_1 a + s_1 b - q(r_2 a + s_2 b)  \\ 
		r &= r_1 a + s_1 b - qr_2 a  -qs_2 b  \\
		r &= (r_1  - qr_2)a + (s_1   -qs_2) b 
\end{align*}

\(r \ge 0\) and  \(r < c\), but if \(r \ne 0\) then \(r \in M\) would be the minimum which is a contradiction to the assumption 
that \(c\) is the minimum, also \(r = 0\), \(k = qc\). Thus, \emph{every element of M is multiplicity of c}. Therefore, 
\(a,b\) are also divisible by \(c\).

Now we need to show that \(c\) is in fact the greatest common divisor. \(c = r_2 a + s_2 b\) 

\[
	c = a \alpha_0 + b \beta_0
\]

Because \(c \mid a \land c \mid b\) we know that \(c \mid a \alpha_0 + b \beta_0 \). We also 
know that \(c \mid \gcd(a,b)\) and thus, \(|c| =|g = \gcd(a,b| \) but because both are positive \(c = g\).

\QED

\subsection{Euclidean Algorithm as Matrix Multiplication}

\textbf{Input:} \(a, b \in \mathbb{Z}\), \(a \neq 0 \neq b\).  

\textbf{Output:} \(t = \gcd(a,b)\) and \(\alpha, \beta \in \mathbb{Z}\) with 
\(\alpha a + \beta b = t\).

\textbf{Algorithm:}

\begin{enumerate}

	\item If \(a = 0\), return \(|b|\), \(\alpha = 0\), and 
    \(\beta = b / |b|\).  
    If \(b = 0\), return \(|a|\), \(\alpha = a / |a|\), and \(\beta = 0\).
    
    \item Compute, for \(r_1 := a\) and \(r_2 := b\), the sequence 
    \((r_n)\) using Algorithm~5.18, i.e.  
    \(r_{n-1} = q_{n-1} r_n + r_{n+1}\) with \(q_{n-1} \in \mathbb{Z}\) 
    and \(r_k = 0\).
    
    \item Define 
    
	\[
		A_n := 
		\begin{bmatrix}
			0 & 1 \\
			1 & -q_{n-1}
		\end{bmatrix}	
	\]

	for \(1 \le n \le k - 2\). Then the first row of the matrix \(A_{k - 2} \cdots A_1\) returns the pair \(\alpha. \beta\)
\end{enumerate}

\subsection{Prime Field}

Given \(p \in \Naturals, p > 1\) is prime, then \(\Integers / p \Integers\) is a field with \(p\) elements. It can also 
be noted \(\Field_p\)

\textbf{Proof:} 

We need to show that for \(a + p \Integers \ne p \Integers\) there exists some \(b + p \Integers\) such that \((a + p\Integers))(b + p\Integers) = 1 + p\Integers\).

For \(a \notin p \Integers\) we know that \(gcd(a,p) = 1\) thus, we do not have zero divisors.

\subsection{Implication of Prime Numbers}

For \(a, b \in \Integers\) and a prime number \(p\)

\[
	p \mid ab \implies (p \mid a ) \lor (p \mid b)
\]

\textbf{Proof:} 

``\(\Rightarrow\)'' Let \(p\) be a prime and \(a,b \in \Integers\)
with \(p \mid ab\). Assume \(p \nmid a\). Then there exist \(\alpha,\beta \in \Integers\)
with \(\alpha p + \beta a = 1\). Since \(p \mid ab\), it follows that
\(p \mid \alpha p b + \beta ab = (\alpha p + \beta a)b = b\).

``\(\Leftarrow\)'' Let \(c \in \Naturals\) be a divisor of \(p\). Then there exists
\(d \in \Naturals\) with \(p = cd\). In particular \(1 \le c \le p\) and
\(1 \le d \le p\) and \(p \mid cd\). By our assumption it follows that
\(p \mid c\) or \(p \mid d\), i.e. \(p \le c\) or \(p \le d\). Together with
\(c \le p\) and \(d \le p\) this forces \(c = 1\) or \(c = p\). Hence, \(p\)
has no nontrivial divisors and is prime.

\QED

\subsection{Fundamental Theorem of Arithmetic}

Every integer greater than 1 can be written in the form

\[
	p_1^{n_1}p_2^{n_2} \cdots p_k^{n_k}
\]

Where \(n_i \geq 0\) and the \(p_i\) are distinct primes. The factorization is unique, except possibly for 
the order of the factors.

\textbf{Example.}

\[
	4312 = 2 \cdot 2156 = 2 \cdot 2 \cdot 1078 = 2 \cdot 2 \cdot 2 \cdot 539 = 2 \cdot 2 \cdot 2 \cdot 7 
	\cdot 77 = 2 \cdot 2 \cdot 2 \cdot 7 \cdot 7 \cdot 11
\]

That is,

\[
	4312 = 2^3 \cdot 7^2 \cdot 11
\]

Approach to the proof: 

\begin{itemize}

	\item The Fundamental Theorem of Arithmetic says that every integer greater than 1 can be factored 
		 uniquely into a product of primes.
	
	\item Euclid’s lemma says that if a prime divides a product of two numbers, it must divide at least 
		  one of the numbers.
	
	\item The least common multiple \([a, b]\) of nonzero integers \(a\) and \(b\) is the smallest positive 
	      integer divisible by both \(a\) and \(b\).

\end{itemize}


\subsection{Lemmas}

\textbf{Lemma} 

If \(m \mid pq\) and \(\gcd(m, p) = 1\), then \(m \mid q\).

\textbf{Proof:} 

Write \(1 = \gcd(m, p) = am + bp\) for some \(a, b \in \Integers\). Then

\[
	q = amq + bpq
\]

Since \(m \mid amq\) and \(m \mid bpq\) (because \(m \mid pq\)), we conclude \(m \mid q\).

\QED

\textbf{Lemma} 

If \(p\) is prime and \(p \mid a_1a_2 \cdots a_n\), then \(p \mid a_i\) for some \(i\).

\textbf{Proof (Case \(n=2\)):} 

Suppose \(p \mid a_1a_2\), and \(p \nmid a_1\).
Then \(\gcd(p, a_1) = 1\), and by the previous lemma, \(p \mid a_2\).

For general \(n > 2\): Assume the result is true for \(n-1\). Suppose \(p \mid a_1a_2 \cdots a_n\).
Group as \((a_1a_2 \cdots a_{n-1})a_n\).

By the \(n=2\) case, either \(p \mid a_n\) or \(p \mid a_1a_2 \cdots a_{n-1}\), and by induction, 
\(p \mid a_i\) for some \(i\).

\QED

\subsection{Proof of the Fundamental Theorem of Arithmetic}

\textbf{Existence:}

Use induction on \(n > 1\)Let \(X = (X_{1}, \ldots, X_{n})^{T} : \Omega \to X\) be a random variable with
statistical model \(P = \{ P^{n}_{\theta} : \theta \in \Theta \}\).
Furthermore, let \(x_{1}, \ldots, x_{n}\) be a sample and let \(\alpha \in [0,1]\).

A mapping \(I\) that assigns to each sample
\(x = (x_{1}, \ldots, x_{n})^{T} \in X\) a subset \(I(x) \subset \Theta\)
is called a \((1 - \alpha)\)-confidence region for \(\theta\) if

\[
    \Prob_{\theta}\bigl( \theta \in I(X_{1}, \ldots, X_{n}) \bigr)
    \ge 1 - \alpha
\]

for all \(\theta \in \Theta\).

If \(\Theta \subset \Reals\) and \(I(x)\) is an interval for all \(x \in X\),
then \(I\) is called a \((1 - \alpha)\)-confidence interval.

Base case: \(n = 2\) is prime.

Inductive step: If \(n\) is prime, done. Otherwise, \(n = ab\), with \(1 < a, b < n\).
By induction, both \(a\) and \(b\) factor into primes, so \(n\) does too.

\textbf{Uniqueness:}

Suppose:

\[
	p_1^{m_1} \cdots p_j^{m_j} = q_1^{n_1} \cdots q_k^{n_k}
\]

With all \(p_i\) and \(q_i\) distinct primes.

Since \(p_1\) divides the LHS, it divides the RHS. So \(p_1 \mid q_i^{n_i}\) for some \(i\), hence 
\(p_1 = q_i\). Re-order so \(p_1 = q_1\). Then:

If \(m_1 > n_1\), divide both sides by \(q_1^{n_1}\):

\[
	p_1^{m_1-n_1} \cdots p_j^{m_j} = q_2^{n_2} \cdots q_k^{n_k}
\]

But then \(p_1\) divides LHS but not RHS, contradiction. So \(m_1 = n_1\). Cancel and repeat.

Eventually, all \(p_i\) match with some \(q_i\), and the exponents are equal. So the factorizations
are the same up to order.

\QED

\subsection{Least Common Multiple}

The least common multiple of \(a\) and \(b\), denoted \([a, b]\), is the smallest positive integer 
divisible by both.

\textbf{Example:}

\[
	[6, 4] = 12, \quad [33, 15] = 165
\]

\subsection{Multiplication Property of the Least Common Multiple}

\[
	[a, b] \cdot \gcd(a, b) = ab
\]

\textbf{Proof:}

Let:

\[
	a = p_1 \cdots p_lq_1 \cdots q_m, \quad b = q_1 \cdots q_mr_1 \cdots r_n
\]

Then:

\begin{align*}
	\gcd(a, b) & = q_1 \cdots q_m                                 \\
	[a, b]     & = p_1 \cdots p_lq_1 \cdots q_mr_1 \cdots r_n     \\
	ab         & = p_1 \cdots p_lq_1^2 \cdots q_m^2r_1 \cdots r_n
\end{align*}

So:

\[
	[a, b] \cdot \gcd(a, b) = ab
\]

\textbf{Example:}

\[
	\gcd(36, 90) = 18, \quad [36, 90] = 180, \quad 36 \cdot 90 = 32400 = 18 \cdot 180
\]


\subsection{Second Proof of the Fundamental Theorem of Arithmetic}

Given \(n \in \Naturals\) has \(1\) as a prime factor. Suppose that every natural number 
smaller than \(n\) can be factorized into primes. 

If \(n\) is a prime number then it is itself its prime factorization. Else. there 
are \(b, c \in \Integers\) with \(n = bc\) \(1 < b, c<n\). By our induction's supposition, 
\(b\) and \(c\) have also a prime factorization.

\[
	b = p_1 \cdot p_2 \cdots p_r \text{ and } c = q_1 \cdot q_2 \cdots q_s
\]

with \(p_i, p_j\) \(1 \le \le r\) and \(1 \le j \le s\) prime numbers. The product is 
the prime factorization of the \(n\).

The uniqueness is given by \emph{the implication of prime numbers}.

\QED

\subsection{Goldbach's Conjecture}

Every even integer greater than 4 can be written as the sum two primes.

\subsection{Prime Factorization}

To find the \emph{prime factorization} of a number \(a\) the algorithm goes as follows 

\begin{itemize}

	\item Let \(n\) be the first prime number. 

	\item  Try dividing \(a\) by \(n\). If \(n \mid a\) note the number and set \(a = a \div n\) to repeat this step. If \(n\) 
	did not divide \(a\) set \(n\) to the next prime number and repeat this step. 

	\item When 1 was reached using the previous step then the algorithm terminates and the noted numbers is our prime factorization 
	of \(a\).

\end{itemize}

\subsection{GCD via Prime factorization}

The \(\gcd\) of two or more number can also be found with the following algorithm.

\begin{itemize}
	
	\item Find the prime factorization of all target numbers.

	\item Note the number which appear in both prime factorization always taking the one with least exponent.
	
	\item Multiply the noted prime factors together. 
	
\end{itemize}

